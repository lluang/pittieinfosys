%\documentclass[12pt, addpoints, answers]{exam}
\documentclass[12pt, letter, addpoints]{exam}
\usepackage{multirow}
\usepackage{pdflscape}
%\usepackage{amssymb}

%\usepackage{makeidx}

%\usiepackage[colorlinks]{hyperref}

% In case we're not using hyperref.sty:
%\providecommand{\texorpdfstring}[2]{#1}
% Thie following can be used in \section commands
% without generating pdf warnings:
%\newcommand{\bs}{\texorpdfstring{\char`\\}{}}

%\newcommand{\docversion}{2.5}
%\newcommand{\docdate}{May 7, 2015}
%\newcommand{\docdate}{\today}

%--------------------------------------------------------------------
%
% Changes since version 2.4 are described in the comments
% near the beginning of the file exam.cls.
%
%--------------------------------------------------------------------

%\makeindex

%\newcommand{\indc}[1]{\index{#1@\texttt{\char`\\#1}}}
%\newcommand{\indcsub}[2]{\index{#1@\texttt{\char`\\#1}!#2}}
%\newcommand{\indcstart}[1]{\index{#1@\texttt{\char`\\#1}|(}}
%\newcommand{\indcstop}[1]{\index{#1@\texttt{\char`\\#1}|)}}

%\newcommand{\indt}[1]{\index{#1@\texttt{#1}}}
%\newcommand{\indtsub}[2]{\index{#1@\texttt{#1}!#2}}
%\newcommand{\indtstart}[1]{\index{#1@\texttt{#1}|(}}
%\newcommand{\indtstop}[1]{\index{#1@\texttt{#1}|)}}

%---------------------------------------------------------------------
\newenvironment{example}%
   {\bigskip\filbreak
    \subsubsection{Example:}
   }%
   {}

\def\samplehead#1#2#3#4{%
  \begin{trivlist}
    \item[]
    \leavevmode
    \hbox to \textwidth{%
      \rlap{\parbox[b]{\textwidth}{\raggedright#1\strut}}%
      \hfil\parbox[b]{\textwidth}{\centering#2\strut}\hfil
      \llap{\parbox[b]{\textwidth}{\raggedleft#3\strut}}%
    }% hbox
    #4
  \end{trivlist}
}

\def\samplefoot#1#2#3#4{%
  \begin{trivlist}
    \item[]
    \leavevmode
    #1
    \vskip 3pt

    \hbox to \textwidth{%
      \rlap{\parbox[t]{\textwidth}{\raggedright#2}}%
      \hfil\parbox[t]{\textwidth}{\centering#3}\hfil
      \llap{\parbox[t]{\textwidth}{\raggedleft#4}}%
    }% hbox
  \end{trivlist}
}
\usepackage{Sweave}
\begin{document}
\input{MidtermSp16A-concordance}

\makebox[\textwidth]{Name:\enspace\hrulefill}

\centering
{\bf IE 0015 Spring 2016 Midterm}

\centering
{\bf A}





\begin{enumerate}
\def\labelenumi{\arabic{enumi}.}
\itemsep1pt\parskip0pt\parsep0pt
\item
  Write your name on the top of this page.
\item
  Exam is closed book with no notes. The article "The Phrase 'Mass Shooting' Belongs To The 21st Century" is provided with the exam. In addition, the last page of the exam includes examples of data from the Gun Violence Archive.
\item
  Answer the questions in the spaces provided on the question sheets. If you run out of room for an answer, continue on the back of the page.
\item
  You have until 1:50 PM to finish. There are 100 points total divided among 3 questions.
\end{enumerate}



\begin{questions}

\question[35]  CoNVO statement

In the Commonwealth of Pennsylvania, the Pennsylvania Department of Education Office for Safe Schools coordinates school safety and security programs and provides technical assistance to school districts regarding school safety. 
Many school districts throughout the Commonwealth have been requesting guidance on staff training for gun related violence. 
Specifically, districts have been asking if they should prioritize training and resources for dealing with students bringing guns to schools or for mass shootings.

\begin{parts}
\part  Write a CoNVO statement for the PA Dept. of Education Office of School Safety for a data analysis motivated the above question.

\end{parts}

\begin{solution}[3in]

- Context - This needs to be an organization that needs to make a decision. The focus should be on the organizations mission that put it into the place that it had to make a decision.  This is important because it gives focus to the analysis that follows.

-  Needs - Has two parts.  A decision that the organization identified in the Context needs to make, and a question that can be answered through data analysis. Both of these need to be specific. The data analysis question in particular needs to be quantitative, the data analysis will show something that can be interpreted.

-  Vision - The vision needs to be a presentation of data that can answer the Need and lead to a decision.  One common mistake is to overly summarize the data.  Summarizing the data hides information, so we only do it after we have determined that a variable is not important.  A sketch (an informal drawing or a mockup) of a chart can help here. Pie charts are especially bad at showing information, bar charts are slightly better.  Pie charts are often used to give justification to a decision already made, but when exploring data it is important to be able to see the variation in it. Because it allows for decisions and courses of action that allows for recognizing that "one size does not fit all."  

-  Outcome - The outcome should be how the data presented in the Vision can answer the question provided in the Need by the organization identified in the Context.  One thing that is important here is that the four parts of the answer are linked together.  One common problem was to answer these four parts independently.
\end{solution}
\pagebreak
\question[35] Data Analysis

As part of an answer on school security you decide to look at all shootings involving school aged children.

\begin{parts}
\part What additional information do you need that is not currently in the records provided by the Gun Violence Archive?
\part Write out a procedure (data manipulation flow) that will enable comparisons between shooting incidents involving children and mass shootings involving children using the three tables from the Gun Violence Archive. Note that many incidents can exist in more than one table.

\begin{solution}[4in]

1.  The data provided does not include the ages of the victims.
2.  Methodology should include the following:
    1.  Combine the child killed and injured and remove duplicates
    2.  Separate mass shooting into with children and without children
    3.  Count mass shootings with and without children
    4.  Compare the two
\end{solution}

\end{parts}
\pagebreak

\begin{center}
\gradetable[v][questions]
\end{center}
\question[30] Data visualization

FedEx Ground is considering adding another distribution center in the southwest Pennsylvania region.  However, like other companies that have large numbers of distribution centers around the country (e.g. UPS, Amazon.com), FedEx Ground finds that finding labor to staff its distribution centers is an issue.  Assume that U.S. Census data and American Community Survey data is available.

\begin{parts}
\part Sketch as data presentation (tables or charts/graphs) that can relate unemployment to the road network (interstate highways).  Include a title and legend/axis labels as appropriate.
\part Write a description of your sketch using visual cues. There should be a description of at least three visual cues in your description. (Note: since it is unlikely that you brought colored pencils, use shading or shapes where you would normally use colors.)
\end{parts}

\begin{solution}[3in]
The solution is likely going to be a map with the road network, and southwest Pennyslvania divided into something like counties. And some charting of unemployment levels (raw numbers) superimposed on the map.  
\end{solution}

\end{questions}
\pagebreak

\begin{landscape}
This exam will make use of the data from the gun violence archive on mass shootings, child injuries, and child deaths.  A sample from each of these tables is below. Note that if an incident involved a child killed and a child injured, that incident would appear in both Table 1 and Table 2.


\centering
{Table 1. Children Killed}
% latex table generated in R 3.2.2 by xtable 1.8-2 package
% Tue Mar  1 23:40:44 2016
\begin{tabular}{llllrr}
  \hline
Incident.Date & State & City.Or.County & Address & X..Killed & X..Injured \\ 
  \hline
February 29, 2016 & Tennessee & Apison & 4724 Gates Lane &   1 &   0 \\ 
  February 26, 2016 & Indiana & Sunman & Van Wedding Road &   1 &   0 \\ 
  February 25, 2016 & North Carolina & Wayne (county) & Sevendales Drive &   1 &   0 \\ 
  February 23, 2016 & Arizona & Glendale & 4812 West Sunnyside Ave &   5 &   0 \\ 
  February 20, 2016 & Florida & Miami & 1225 NE 103rd Lane &   1 &   0 \\ 
  February 17, 2016 & Indiana & Zionsville & 6833 Old Hunt Club Rd &   3 &   0 \\ 
  February 16, 2016 & Texas & Lubbock & 2500 block of  71st Street &   1 &   0 \\ 
   \hline
\end{tabular}

\centering
{Table 2. Children Injured}

% latex table generated in R 3.2.2 by xtable 1.8-2 package
% Tue Mar  1 23:40:44 2016
\begin{tabular}{llllrr}
  \hline
Incident.Date & State & City.Or.County & Address & X..Killed & X..Injured \\ 
  \hline
March 1, 2016 & Tennessee & Nashville & 1500 block of Porter Road &   0 &   3 \\ 
  February 29, 2016 & New Jersey & Paterson & 43 Rosa Parks Boulevard &   0 &   1 \\ 
  February 29, 2016 & South Carolina & Spartanburg & 340 Rowe St. &   0 &   1 \\ 
  February 28, 2016 & Virginia & Portsmouth & 4200 block of Moregate Lane &   0 &   1 \\ 
  February 28, 2016 & Virginia & Portsmouth & 700 block of Lancer Drive &   0 &   1 \\ 
  February 27, 2016 & New York & Buffalo & N/A &   0 &   2 \\ 
  February 27, 2016 & California & Mc Farland & N/A &   0 &   2 \\ 
   \hline
\end{tabular}

\centering
{Table 3. Mass Shootings}

% latex table generated in R 3.2.2 by xtable 1.8-2 package
% Tue Mar  1 23:40:44 2016
\begin{tabular}{llllrr}
  \hline
Incident.Date & State & City.Or.County & Address & X..Killed & X..Injured \\ 
  \hline
February 28, 2016 & Michigan & Detroit & 8800 block of Livernois &   0 &   5 \\ 
  February 28, 2016 & California & Riverside (Jurupa Valley) & 5500 block of Etiwanda Avenue &   1 &   3 \\ 
  February 27, 2016 & Virginia & Woodbridge (Lake Ridge) & 13051 Lashmere Court &   2 &   2 \\ 
  February 26, 2016 & Washington & Belfair & 300 block of Horseshoe Rd. &   5 &   0 \\ 
  February 25, 2016 & Kansas & Hesston & 200 S Ridge Rd &   4 &  14 \\ 
  February 23, 2016 & Arizona & Glendale & 4812 West Sunnyside Ave &   5 &   0 \\ 
  February 23, 2016 & Florida & Daytona Beach & 1042 Sheridan Road &   0 &   4 \\ 
   \hline
\end{tabular}
\end{landscape}

\pagebreak
\end{document}
